\documentclass[12pt]{article}
\usepackage[left=0.8in,right=0.8in,top=2cm,bottom=3cm]{geometry}

\usepackage{newpxtext,newpxmath}




\makeatletter
\makeatother





\begin{document}
	

\begin{center}
	\textbf{Cover Letter}
\end{center}


\noindent Dear USEQIP and URA coordinators, \\



My name is Huan Bui. I am currently in my third year at Colby College studying physics, mathematic, and statistics, and I wish to participate in the USEQIP and URA at the Institute for Quantum Computing (IQC) this summer. I am drawn to working on problems in quantum information (QI) and atomic, molecular, and optical (AMO) physics research, for they often involve a strong interplay between theories and experiments. New theories are readily tested with table-top experiments equipped with exceptional abilities to coherently control quantum systems, while novel results from AMO/QI experiments constantly pose new sproblems for theorists. I wish to pursue experimental quantum information research with an equally strong emphasis on theories and their applications. To this end, I have been taking advanced physics and mathematics courses and actively involved in experimental AMO physics research at Colby (2017\textendash present) and at the Joint Quantum Institute (JQI) in College Park, MD (summer \& winter 2019). I am also recently involved in applied mathematics research at my college. My interest in QI stems from my lab work and my desire to apply mathematics to describe quantum systems. \\

This past summer at JQI, I joined Professor Steven Rolston's group where we study infinite-range interactions of cold rubidium atoms trapped around an optical nanofiber (ONF) in a magneto-optical trap (MOT) via measuring their collective decay. One of our future endeavors is to have an optical dipole trap using nanofibers. However, unlike in typical optical dipole traps where no waveguide is used, a control system for light polarization state is necessary in our setup to account for the non-uniform birefringence and cylindrically asymmetric longitudinal polarization introduced by the nanofiber waist and the fiber tapered region. To this end, I built the Nd:YAG 1064 nm optical arrangement for the dipole trap light and created a method to optimize laser light polarization at the tapered region of the ONF. Under the supervision of Dr. Hyok Sang Han, I was able to obtain quasi-linearly polarized light via waveplates, polarization controllers, and an imaging system which I created to quantify quasi-circular and quasi-elliptical polarizations. The imaging system consists of two orthogonal CCD cameras, equipped with a polarization filter, from which the ratio of detected optical power characterizes the polarization state of light in the ONF waist region of interest. During my time at JQI, I also developed a complete, stand-alone experimental control program in Python using the NI-DAQmx libraries to replace the less reliable program in LabView. \\ 

I attribute my productivity at JQI to the experience I have gotten from more than two years of experimental AMO physics research ultracold Rydberg potassium atoms at Colby College with Professor Charles Conover. During this time, I learned the basics of AMO physics research through hands-on work and involvement in most experiments and explorations. I have built dither and frequency-locking electronics to control and stabilize external cavity lasers that are used to create Rydberg atoms in the lab. I have operated and programmed various arbitrary waveform generators/power supplies for a number of purposes, including fast MOT field switching to the behavior of the MOT cloud in the abrupt absence of the magnetic field. I am also extremely familiar with operating the MOT and external-cavity diode lasers to create Rydberg atoms in the lab. Our on-going projects include atomic spectroscopy on Rydberg potassium, electromagnetically-induced transparency, and measuring radiative lifetimes of potassium $5p$ and $4p$ states in a MOT. \\


In previous years, the Conover group focused primarily on precision measurements of $d$-$d$ and $s$-$p$ Rydberg millimeter-wave transitions in potassium. Our precision measurement procedure consisted of correcting for dc-Stark shifts by applying static electric fields and correcting for ac-Stark shifts by making measurements at a range of millimeter-wave powers and extract the zero-power value. Apart from requiring data extrapolation, this second step was also limiting because not every Rydberg-Rydberg transition was strong across a range of millimeter-wave power. My role over the summer following my freshmen year was to study Ramsey's separated oscillatory fields method as an alternative to our conventional 3-step method. I modified the single-pulse excitation scheme to a double-pulse scheme and recorded Ramsey fringes in the frequency domain and Rabi oscillations in the time domain. From there, I used a simple two-level Hamiltonian model to derive mathematical expressions for the observed fringes and oscillations, from which I extracted the zero-ac-Stark shift transition frequency from a single measurement at which the signal was strong. This work resulted in a poster presentation at my college's summer research retreat (CUSRR 2018) and another at APS DAMOP 2019.  \\





Beside experimental work in the labs,  I am also fascinated by theoretical physics and the applications of mathematics in quantum information and atomic physics. I have been exploring these topics by taking advanced courses and independent studies. By the end of this academic year, I will have finished the required physics curriculum plus a semester of quantum information and four on classical field theory-massive gravity (material from Hinterbichler\cite{RevModPhys.84.671}, Carroll's \textit{Spacetime \& Geometry}, and Zee's \textit{Quantum Field Theory in a Nutshell}). I will also have completed two semesters of linear algebra, abstract algebra (with algebraic geometry), analysis, probability, and differential equations. For my matrix analysis final project last spring, I presented the construction of the tensor product and its application in representing joint quantum states in a simple 2-qubit entanglement circuit. Now, I am studying convolution powers of complex-valued functions with Professor Evan Randles. I am hoping to turn my results into an Honors Thesis for the mathematics major. \\

  
A summer research at the Institute for Quantum Computing (IQC) offers me an opportunity to apply my experience in experimental AMO physics and interest in theory to tackle problems in quantum information research. I have contacted Professor Crystal Senko, whose research on the theoretical aspects and experimental implementation of qudits align exactly with my research interests. We have also discussed possible projects for the summer, many of which I find very enticing. As I wish to pursue quantum information research in graduate school and academia after my undergraduate studies, I believe a USEQIP and a URA at the IQC will give me the chance to found a strong transition.\\

\noindent Thank you for your time and consideration. \\



\noindent Best regards,\\
\noindent Huan Q. Bui













\bibliography{ref} 
\bibliographystyle{ieeetr}








	
	
	
	
	
\end{document}