\documentclass[10pt]{article}
\usepackage[utf8]{inputenc}
\usepackage{inconsolata}
\usepackage{setspace}
\usepackage{graphicx}
\usepackage[left=1in,right=1in,top=0.7in,bottom=1.0in]{geometry}
\pagenumbering{gobble}
\usepackage{xcolor}
\definecolor{colby}{rgb}{0.0, 0.0, 0.5}
\usepackage[pdftex]{hyperref}
\hypersetup{colorlinks,urlcolor=colby}
\usepackage[normalem]{ulem}
\usepackage{booktabs}% http://ctan.org/pkg/booktabs
\newcommand{\tabitem}{~~\llap{\textbullet}~~}
\newcommand{\longunderline}[1]{\uline{#1\hfill\mbox{}}}
\usepackage{array}

\usepackage{longtable}



\usepackage{newpxtext,newpxmath}



\pagenumbering{gobble}



\makeatletter
\makeatother





\begin{document}
	

\noindent  





\begin{center}
	\Huge{Huan Q. Bui} 
\end{center}
\vspace{-7pt}
\begin{center}
	\normalfont{8347 Mayflower Hill, Waterville, ME, 04901 $ \bullet $ 301-704-6958 $ \bullet $ \href{mailto:hqbui21@colby.edu}{\texttt{hqbui21@colby.edu}}}
\end{center}








	
%	
%\begin{center}
%	\begin{tabular}{l l}
%		8347 Mayflower Hill 		 & \hspace{1in} Email: \href{mailto:hqbui21@colby.edu}{\texttt{hqbui21@colby.edu}} \\
%		Colby College 				 & \hspace{1in}  Websites: \href{www.huanqbui.com}{\texttt{huanqbui.com}} $\vert$ 					\href{https://www.linkedin.com/in/huan-bui/}{\includegraphics[scale=0.04]{linkedin_logo.PNG}} $\vert$ \href{https://github.com/huanium/huanium}{\includegraphics[scale=0.02]{GitHub-Mark.PNG}}\\
%		Waterville, Maine, USA 04901 & \hspace{1in} Phone: +1 (301)-704-6958\\
%	\end{tabular}
%\end{center}
%\vspace{-10pt}




\noindent March 13, 2020\\
\noindent IBM Quantum Computing Developer Coordinator\\
\noindent 1101 Kitchawan Rd, \\
\noindent Yorktown Heights, NY 10598\\






\noindent Dear Coordinator, \\


\noindent I am writing to apply for the IBM Quantum Computing Developer Intern position at IBM Yorktown Heights, NY. I am drawn to problems in quantum information and quantum computing (QI/QC), for they involve a strong interplay between theories and experiments. As I wish to pursue QI/QC research in graduate school, I believe this opportunity will be an excellent experience for me.\\ 
%I am also interested in general relativity and mathematical physics. 
%I am currently involved in applied mathematics research which has applications in quantum theories.   \\


\noindent I have been actively involved in various physics and mathematics research projects at Colby College and at the Joint Quantum Institute (JQI) in College Park, MD. At JQI, I joined the Rolston Group where we study infinite-range interactions of Rb atoms trapped around an optical nanofiber (ONF) via their collective decay. 
%One of our future endeavors is to have an optical dipole trap using an ONF. 
%However, unlike in typical free-space dipole traps where no waveguide is used, a control system for light polarization state is necessary in our setup to account for non-uniform birefringence and a cylindrically asymmetric longitudinal polarization state introduced by the ONF. 
I built the Nd:YAG 1064 nm optical arrangement for an optical dipole trap and created a method to optimize light polarization in the ONF. I also developed a stand-alone experimental control program in Python using the NI-DAQmx libraries. In January 2020, I became directly involved in the collective decay measurements.\\ 

\noindent I attribute my opportunity at JQI to more than two years of experience researching Rydberg K atoms at Colby with Professor Conover. In his lab, I built electronics to stabilize external cavity diode lasers' wavelengths and programmed waveform generators for various purposes. Over the summer of 2018, I studied Ramsey's separated oscillatory fields as an alternative to our conventional three-step precision measurement method (accurate up to $5$ parts in $10^8$). I used a simple two-level atom model to derive mathematical expressions for the observed Ramsey fringes, from which I extracted the desired measurements with only two steps. This work resulted in a poster presentation at my college's research retreat and another at APS DAMOP 2019.  \\


\noindent Beside experimental work, I am exploring theoretical physics and applied mathematics in QI/AMO physics and general relativity. By the end of this academic year, I will have finished the required physics curriculum plus one semester on QI/QC and four semesters on classical field theory and massive gravity (material from Hinterbichler  (RevModPhys.84.671), Carroll's \textit{Spacetime \& Geometry}, and Zee's \textit{Quantum Field Theory in a Nutshell}). I will also have completed two semesters of linear algebra, abstract algebra, analysis, probability, and differential equations. For my Matrix Analysis final project of Spring 2018, I presented the construction of the tensor product and its application in a simple 2-qubit entanglement quantum circuit. Now, I am researching the convolution powers of complex-valued functions with Professor Evan Randles. I hope to turn my results into an Honors Thesis in mathematics. \\

  
\noindent A summer research on quantum computing at IBM will provide me with an excellent opportunity to bring my experience in experimental physics and interest in theory to tackling problems in quantum computing.  I believe this opportunity will be an exceptional transition to graduate school and academia.\\

\noindent Thank you for your consideration. \\

\noindent Sincerely,\\

$\,$\\
$\,$\\


\noindent Huan Q. Bui



%\bibliography{ref} 
%\bibliographystyle{ieeetr}
















	
	
	
	
	
\end{document}