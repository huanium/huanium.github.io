\documentclass[10pt]{article}
\usepackage[left=0.8in,right=0.8in,top=2cm,bottom=2cm]{geometry}

\usepackage{newpxtext,newpxmath}



\pagenumbering{gobble}



\makeatletter
\makeatother





\begin{document}
	
	
\begin{center}
	\textbf{Statement of Purpose}
\end{center}
	

\noindent I am a third-year undergraduate at Colby College majoring in physics and mathematics and minoring in statistics, and I am applying for the both the summer school and a summer research internship at the Perimeter Institute. I am drawn to problems in quantum information (QI) and atomic, molecular, and optical (AMO) physics research, for they involve a strong interplay between theories and experiments. As I wish to pursue theoretical QI research in graduate school, I believe an opportunity at the Perimeter Institute will be an excellent experience for me. My interest in QI stems from my lab work and desire to describe quantum systems with mathematics. I have been taking advanced physics and mathematics courses and actively involved in AMO physics research at Colby (since 2017) and at the Joint Quantum Institute (JQI) at the University of Maryland, College Park (summer, winter 2019). I am also interested in general relativity and mathematical physics. I am currently involved in applied mathematics research and independent studies on massive gravity at Colby.   \\


At JQI, I joined the Rolston Group where we study infinite-range interactions of Rb atoms trapped around an optical nanofiber (ONF) via their collective decay. One of our future endeavors is to have an optical dipole trap using an ONF. However, unlike in typical free-space dipole traps where no waveguide is used, a control system for light polarization state is necessary in our setup to account for non-uniform birefringence and a cylindrically asymmetric longitudinal polarization state introduced by the ONF. My contribution was building the Nd:YAG 1064 nm optical arrangement and creating a method to optimize polarization in the ONF. I was able to obtain quasi-linearly polarized light via an imaging system, which I created to quantify circular and elliptical polarizations. The system consists of two orthogonal CCD cameras equipped with polarizing filters from which the ratio of detected optical power characterizes the polarization state in the ONF. I also developed a stand-alone experimental control program in Python using the NI-DAQmx libraries, independent of LabView. At the moment, I am directly involved in the collective decay measurements. \\ 

I attribute my opportunity at JQI to more than two years of experience researching Rydberg K atoms at Colby with Professor  Conover. In his lab, I have built electronics to stabilize external cavity diode lasers' wavelengths and programmed waveform generators for various purposes including fast MOT field switching to study the dynamics of the MOT cloud in the abrupt absence of the trapping field. In previous years, the Conover Group focused on precision measurements of $d$-$d$ and $s$-$p$ Rydberg mm-wave transitions in K. My role over the summer of 2018 was to study Ramsey's separated oscillatory fields as an alternative to our conventional three-step measurement method. I modified the single-pulse excitation scheme to double-pulse and recorded Ramsey fringes in the frequency domain and Rabi oscillations in the time domain. From there, I used a simple two-level atom model to derive mathematical expressions for the observed fringes and oscillations, from which I extracted the desired measurements with only two steps. This work resulted in a poster presentation at my college's research retreat (CUSRR 2018) and another at APS DAMOP 2019.  \\


Beside experimental work, I am fascinated by theoretical physics and the applications of mathematics in QI/AMO physics and general relativity, which I have been exploring in advanced courses and independent studies. By the end of this academic year, I will have finished the required physics curriculum plus one semester on QI and four semesters on classical field theory and massive gravity (material from Hinterbichler\cite{RevModPhys.84.671}, Sean Carroll's \textit{Spacetime \& Geometry}, and Anthony Zee's \textit{Quantum Field Theory in a Nutshell}). I will also have completed two semesters of linear algebra, abstract algebra (with algebraic geometry), analysis, probability, and differential equations. For my Matrix Analysis final project of Spring 2018, I presented the construction of the tensor product and its application in a simple 2-qubit entanglement quantum circuit. Now, I am researching the convolution powers of complex-valued functions and related topics in harmonic analysis with Professor Evan Randles. I hope to turn my results into an Honors Thesis for the mathematics major. \\

  
A summer research at the Perimeter Institute will provide me with an excellent opportunity to apply my experience in experimental physics and interest in theory at large to tackle problems in mathematical or theoretical physics. As I wish to pursue physics in graduate school and academia, I believe a summer research at Perimeter will allow me to establish a strong transition.










\bibliography{ref} 
\bibliographystyle{ieeetr}
















	
	
	
	
	
\end{document}