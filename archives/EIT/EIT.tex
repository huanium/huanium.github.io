\documentclass{article}
\usepackage{physics}
\usepackage{graphicx}
\usepackage{caption}
\usepackage{amsmath}
\usepackage{authblk}
\usepackage{amsfonts}
\usepackage{esint}
\usepackage{mathtools}
\usepackage{amsthm}
\theoremstyle{definition}
\newtheorem{defn}{Definition}[section]
\newtheorem{prop}{Proposition}[section]
\newtheorem{rmk}{Remark}[section]
\newtheorem{exmp}{Example}[section]
\usepackage{empheq}
\usepackage{hyperref}
\usepackage{tensor}
\usepackage{xcolor}
\hypersetup{
	colorlinks,
	linkcolor={black!50!black},
	citecolor={blue!50!black},
	urlcolor={blue!80!black}
}
\usepackage[english]{babel}
\usepackage[utf8]{inputenc}
\usepackage{fancyhdr}

\pagestyle{fancy}
\fancyhf{}
\rhead{Huan Q. Bui}
\lhead{Electromagnetically Induced Transparency}
\rfoot{\thepage}

\begin{document}
\section{Introduction}
The following introduction to Electromagnetically Induced Transparency (EIT) is based on Stephen E. Harris' article in \textit{PHYSICS TODAY}, 1997. \href{https://web.stanford.edu/group/harrisgroup/PAPERS/review.pdf}{\underline{Link}}, and various other sources including this well-written \href{https://www.reed.edu/physics/faculty/illing/campus/pdf/FurmanThesis16.pdf}{\underline{thesis}} by Furman of Reed College, this \href{https://www.kip.uni-heidelberg.de/kw/image/f/group/f17/files/F65_EIT_Anleitung.pdf}{paper}, and of course \textit{Principles of Laser Spectroscopy and Quantum Optics} by Paul Berman.\\

In shortest possible terms, EIT is a technique which renders an otherwise opaque atomic medium transparent with electromagnetic radiation. The medium is typically a \href{http://community.dur.ac.uk/thomas.billam/JQC_Atom_Light_2015-2016_L7.pdf}{\underline{three-level}} atomic system, with specific requirements: two of the possible three transitions must be dipole-allowed (so transition rules satisfied) and one not dipole-allowed. The spectrum of the medium without (blue) and with (red) EIT is shown below:
\begin{figure}[h!]
\centering
\includegraphics[scale=0.5]{EIT_spectrum.jpg}
\caption{}
\end{figure}

Notice how there is no absorption at resonance frequency with EIT. Typically, in the presence of a near-resonance field, a two-level atom with ground state $\ket{1}$ and excited state $\ket{2}$ will interact with the field, resulting in a non-zero probability amplitude of the excited state, $\vert P_2\vert^2 = \bra{2}\ket{2} > 0$. If $\vert P_2(\delta\omega) \vert^2$ is the population of the excited state as a function of the detuning, then it follows the blue, Lorentzian line in the above figure. In EIT where there are two radiation fields, though, the energy levels of a three-level atom are altered. This in turn creates a window of frequencies at which the medium is transparent. \\

\section{Derivation: Static Dark State}

The following derivation is (heavily) inspired by Furman's derivation and the Heidelberg paper, but roughly condensed and sprinkled with a bit of my own narratives and insights here and there. \\

Let's consider a $\Lambda$ configuration:
\begin{figure}[h!]
	\centering
	\includegraphics[scale=0.5]{Lambda.PNG}
	\caption{}
\end{figure}

Let the energy of each state $\ket{n}$, $n=\{1,2,3\}$ be
\begin{align}
E_n = \bra{i}\hat{H}\ket{n} = \hbar\omega_n.
\end{align}
where $\hat{H}$ is the neutral atom Hamiltonian. Assume that the transition $\ket{1} \rightarrow \ket{2}$ is forbidden (just as shown in Figure 2). Since $E_n$ and $\ket{n}$ are the eigenvalues and eigenstates of $\hat{H}$, respectively, we let the bare-atom Hamiltonian $\hat{H_0}$ be:
\begin{align}
\hat{H_0}=
\hbar\begin{pmatrix}
\omega_1 & 0 & 0\\
0 & \omega_2 & 0\\
0 & 0 & \omega_3
\end{pmatrix}.
\end{align}
We can do this because the eigenstates $\ket{n}$ form an orthonormal basis. Next, let the applied fields be 
\begin{align}
\vec{E} &= \vec{E}_p \cos(\omega_p t - \vec{k_p}\cdot\vec{r}) +
\vec{E}_c \cos(\omega_c t - \vec{k_c}\cdot\vec{r}) \nonumber\\
&\approx \vec{E}_p \cos(\omega_p t) +
\vec{E}_c \cos(\omega_c t)\nonumber\\
&= \frac{\vec{E}_p}{2}\left(e^{i\omega_pt} - e^{-i\omega_pt} \right) + \frac{\vec{E}_c}{2}\left( e^{i\omega_ct} + e^{i\omega_ct}\right). 
\end{align}
where $\omega_p \approx \omega_{13} = \omega_3 - \omega_1$, and $\omega_c \approx \omega_{23} = \omega_3 - \omega_2$. The subscripts $p$ and $c$ means \textbf{probe} and \textbf{coupling}, respectively. We should also denote the relevant detuning $\delta_p = \omega_{13} - \omega_p$ and $\delta_c = \omega_{12} - \omega_c$. It makes sense to label our subscripts this way, because in the end we are interested in the probability amplitude of $\ket{3}$ as a function of the detuning $\delta_p$ of the probe beam from (bare atom) resonance.\\

With the perturbation from the radiation, the new Hamiltonian of the atom is:
\begin{align}
\hat{H} = \hat{H}_0 + \hat{H}_1,
\end{align}
where, with $\hat{\rho} \equiv qd$ being the dipole moment operator:
\begin{align}
\hat{H}_{1,ij} = -E\hat{\rho}_{ij},
\end{align}
where $E$ is the field strength. We will see how $E$ relates to the Rabi rate $\Omega$ later. Now, a state does not dipole-interact with itself, hence $\rho_{ii} = 0$. In addition, since the transition $\ket{1} \rightarrow \ket{2}$ is forbidden, $\rho_{12} = \rho{21} = 0$. It follows that:
\begin{align}
\hat{H} &= \hat{H}_0 + \hat{H}_1 \nonumber\\
&= \hbar\begin{pmatrix}
\omega_1 & 0 & 0\\
0 & \omega_2 & 0\\
0 & 0 & \omega_3
\end{pmatrix} -
E\begin{pmatrix}
0 & 0 & \rho_{13}\\
0 & 0 & \rho{23}\\
\rho_{31} & \rho_{32} & 0
\end{pmatrix}.
\end{align}
Now, what we have been working with so far are the time-independent eigenstates $\ket{n}$. In the following steps we shall bring in the time-dependent parts. To do this, we invoke the unitary matrix $U(t)$, which transforms $\ket{n}$ into full time-dependent wavefunctions:
\begin{align}
U(t) = e^{iH_0t/\hbar}
\begin{pmatrix}
e^{i\omega_1t} & 0 & 0\\
0 & e^{i\omega_2t} & 0\\
0 & 0 & e^{i\omega_3t}
\end{pmatrix}.
\end{align} 
Obviously, if we apply $\hat{U}(t)$ to $\hat{H}_0$ there wouldn't be anything interesting, since $\hat{U}(t) = d/dt\hat{H}_0$. But we can apply $\hat{U}(t)$ to $\hat{H}_1$. The change of basis rule gives
\begin{align}
\hat{U}(t)\hat{H}_1\hat{U}^{\dagger}(t) = 
-E\begin{pmatrix}
0 & 0 & \rho_{13}e^{-i\omega_{13}t}\\
0 & 0 & \rho_{23}e^{-i\omega_{23}t}\\
\rho_{31}e^{i\omega_{13}t} & \rho_{32}e^{i\omega_{32}t} & 0
\end{pmatrix}.
\end{align}
Multiplying $E$ into $\hat{H}_1$ and applying rotating wave approximation to result gives the non-zero matrix elements:
\begin{align}
\left( \hat{U}(t)\hat{H}_1\hat{U}^\dagger\right)_{13} &= 
-\frac{1}{2}E_p \rho_{13}e^{i(\omega_p - \omega_{13})t}\\
\left( \hat{U}(t)\hat{H}_1\hat{U}^\dagger\right)_{23} &= 
-\frac{1}{2}E_c \rho_{23}e^{i(\omega_c - \omega_{23})t}\\
\left( \hat{U}(t)\hat{H}_1\hat{U}^\dagger\right)_{31} &= 
-\frac{1}{2}E_p \rho_{31}e^{i(\omega_p - \omega_{31})t}\\
\left( \hat{U}(t)\hat{H}_1\hat{U}^\dagger\right)_{32} &= 
-\frac{1}{2}E_c \rho_{32}e^{i(\omega_c - \omega_{32})t}
\end{align}
Transforming $\hat{H}_1$ back to the vector space of $\ket{n}$ gives:
\begin{align}
\hat{H_1} &= \hat{U}^\dagger(t) \left( \hat{U}(t) \hat{H}_1 \hat{U}^\dagger(t)\right) \hat{U}(t) \nonumber \\
&= 
-\frac{1}{2}\begin{pmatrix}
0 & 0 & E_p\rho_{13}e^{i\omega_pt}\\
0 & 0 & E_c\rho_{23}e^{i\omega_ct}\\
E_p\rho_{31}e^{-i\omega_pt} & E_c\rho_{32}e^{-i\omega_ct} & 0
\end{pmatrix}.
\end{align}
Now, to put the dipole moment in terms of the Rabi frequency $\Omega$:
\begin{align}
\Omega_p = \frac{E_p\vert \rho_{13} \vert}{\hbar} &= \frac{E_p\rho_{13}e^{-i\phi_p}}{\hbar}\\
\Omega_c = \frac{E_c\vert \rho_{23} \vert}{\hbar} &= \frac{E_c\rho_{23}e^{-i\phi_c}}{\hbar}
\end{align}
So the total Hamiltonian is
\begin{align}
\hat{H} = \frac{\hbar}{2}
\begin{pmatrix}
2\omega_1 & 0 & -\Omega_pe^{i(\omega_pt+\phi_p)}\\
0 & 2\omega_2 & -\Omega_ce^{i(\omega_ct+\phi_c)}\\
-\Omega_pe^{-i(\omega_pt+\phi_p)} & -\Omega_ce^{-i(\omega_ct+\phi_c)} & 2\omega_3
\end{pmatrix}
\end{align}
Now, since we want our final result in terms of the frequencies only, we shall express $\hat{H}$ in a new basis that is time and phase independent. Let a unitary matrix $\tilde{U}(t)$ be given such that it satisfies the required change of basis. It follows that the eigenstates transform as
\begin{align}
\ket{\tilde{n}} = \tilde{U}(t)\ket{n'}.
\end{align}
Let the new Hamiltonian in this new basis be $\tilde{H}$. The new eigenstate also has to solve the Schr\"{o}dinger equation in the new basis, so
\begin{align}
\tilde{H}\ket{\tilde{n}} &= i\hbar \frac{\partial }{\partial t}\ket{\tilde{n}}\nonumber \\
&= i\hbar\frac{\partial}{\partial t}\left(\tilde{U}\ket{n'} \right) \nonumber\\
&= i\hbar\left( \frac{\partial \tilde{U}}{\partial t}\ket{n'} + \tilde{U}\frac{\partial}{\partial t}\ket{n'}\right)\nonumber\\
&=  i\hbar\left( \frac{\partial \tilde{U}}{\partial t}\ket{n'} + \frac{1}{i\hbar}\tilde{U}\hat{H}\ket{n'}\right) \nonumber\\
&= \left(i\hbar \frac{\partial \tilde{U}}{\partial t}\tilde{U}^\dagger + \tilde{U}\hat{H}\tilde{U}^\dagger \right) \tilde{U}\ket{n'}\nonumber\\
&= \left(i\hbar \frac{\partial \tilde{U}}{\partial t}\tilde{U}^\dagger + \tilde{U}\hat{H}\tilde{U}^\dagger \right) \ket{\tilde{n}}.
\end{align}
So $\tilde{H}$ can be readily computed:
\begin{align}
\tilde{H} &= \left(i\hbar \frac{\partial \tilde{U}}{\partial t}\tilde{U}^\dagger + \tilde{U}\hat{H}\tilde{U}^\dagger \right)\nonumber\\
&= \frac{\hbar}{2}\begin{pmatrix}
2(\omega_1 + \omega_p) & 0 & \Omega_p\\
0 & 2(\omega_2+\omega_c) & -\Omega_c\\
-\Omega_p & -\Omega_c & 2\omega_3
\end{pmatrix}
\end{align}
Note that we can add $\hbar(\omega_1+\omega_p)\hat{I}$ to $\tilde{H}$ without changing the physical interpretation. In fact, the form of $\tilde{H}$ now is a result of our definition of the $\omega$'s. Now, let us define the detunings $\delta_c = \omega_{23}-\omega_c = \omega_3 - \omega_2 - \omega_c $ and $\delta_p = \omega_{13} - \omega_p = \omega_3 - \omega_1 - \omega_p$. So, the ``better'' Hamiltonian is
\begin{align}
\tilde{H}' &= \tilde{H} - \hbar(\omega_1+\omega_p)\hat{I}\nonumber\\
&= \hbar\begin{pmatrix}
0 & 0 & -\Omega_p\\
0 & 2(\delta_p - \delta_c) & -\Omega_c\\
-\Omega_p & -\Omega_c & 2\delta_p
\end{pmatrix}
\end{align}
Assuming $\delta_p \approx \delta_c$, the eigenvalues of $\tilde{H}'$ are:
\begin{align}
E^+ &= \hbar\omega^+ = \frac{\hbar}{2}\left( \delta_p + \sqrt{\delta^2_p + \Omega^2_p + \Omega^2_c}\right) \nonumber\\
E^- &= \hbar\omega^- = \frac{\hbar}{2}\left( \delta_p - \sqrt{\delta^2_p + \Omega^2_p + \Omega^2_c}\right) \nonumber\\
E^0 &= \hbar\omega = 0.
\end{align}
And the eigenstates are:
\begin{align}
\ket{a^0} &= \frac{\Omega_c\ket{1} - \Omega_c\ket{2}}{\sqrt{\Omega^2_p + \Omega^2_c}} = \cos\theta\ket{1} - \sin\theta\ket{2}\nonumber\\
\ket{a^-} &= -\frac{\Omega_p\ket{1} + \Omega_c\ket{2}}{\delta_p - \sqrt{\delta^2_p + \Omega^2_p + \Omega^2_c}} +\ket{3} = \sin\theta\cos\phi\ket{1} + \cos\theta \cos \phi \ket{2} - \sin\phi\ket{3}\nonumber\\
\ket{a^+} &= +\frac{\Omega_p\ket{1} + \Omega_c\ket{2}}{\delta_p + \sqrt{\delta^2_p + \Omega^2_p + \Omega^2_c}} -\ket{3} = \sin\theta\cos\phi\ket{1} + \cos\theta \cos \phi \ket{2} + \cos\phi\ket{3}
\end{align}
where 
\begin{align}
\tan\theta &= \frac{\Omega_p}{\Omega_c}\\
\tan\phi &= \frac{\sqrt{\Omega^2_p+\Omega^2_c}}{\delta_p +  \sqrt{\delta^2_p + \Omega^2_p + \Omega^2_c}}.
\end{align}
Notice that the transition probability of $\ket{a^0} \rightarrow \ket{3}$ is 0, because:
\begin{align}
\bra{a^0}\hat{H}\ket{3} = 0.
\end{align}
We can also see this because $\ket{a^0}$ is not expressed in terms of $\ket{3}$, i.e., it does not have $\ket{3}$ component. We call $\ket{a^0}$ the \textit{dark state} because the transition to $\ket{3}$ is not possible.\\

Now, in the case where $\Omega_p \ll \Omega_c$, i.e., the probe beam is much weaker than the coupling beam, we get
\begin{align}
\sin\theta \rightarrow 0, \cos\theta \rightarrow 1, \ket{a^0} \rightarrow \ket{1},
\end{align}
which means that $\ket{1}$ becomes a dark state, hence this opens a window of frequency where $\ket{1}$ becomes transparent to previously resonance frequencies. This is exactly electromagnetically induced transparency.


\section{Derivation: Dynamic EIT}
We have derived and explained how $\ket{1}$ becomes a dark state when $\Omega_p \ll \Omega_c$. Now, in order to derive the absorption profile, we need a dynamic description of the three-level system where we also take into account spontaneous emission. We do this using the \textit{density matrix formalism}. Let $r$ be the density operator, defined as:
\begin{align}
r = \sum_{n}P_n\ket{n}\bra{n},
\end{align}
where $P_i$ is the probability of finding the atom in state $i$. Note that the definition is simply describing a statistical mixture of the $n$ states. If given an operator $\mathcal{O}$, the expectation value of the measurement of a statistical mixture of states is given by:
\begin{align}
\langle \mathcal{O}\rangle &=\sum_{n}P_n\bra{n}\mathcal{O}\ket{n}\nonumber\\
&= \sum_{n}P_n\tr(\ket{n}\bra{n}\mathcal{O})\nonumber\\
&= \sum_{n}\tr(P_n\ket{n}\bra{n}\mathcal{O})\nonumber\\
&= \tr\left( \sum_{n}P_n\ket{n}\bra{n}\mathcal{O}\right) \nonumber\\
&= \tr(r\mathcal{O}).
\end{align}
Another way to derive the above equality is given in Furman's thesis:
\begin{align}
\langle \mathcal{O}\rangle &=\sum_{n}P_n\bra{n}\mathcal{O}\ket{n}\nonumber\\
&= \sum_{m,n}P_n\bra{n}\mathcal{O}\ket{m}\bra{m}\ket{n}\nonumber\\
&= \sum_{m,n}\bra{m}P_n\ket{n}\bra{n}\mathcal{O}\ket{m}\nonumber\\
&= \sum_{m}\bra{m}r\mathcal{O}\ket{m}\nonumber\\
&= \tr(r\mathcal{O}).
\end{align}
Note that for pure states:
\begin{align}
\langle \mathcal{O} \rangle = \bra{n}\mathcal{O}\ket{n}.
\end{align}
To relate $r$ to the Hamiltonian $\hat{H}$, we invoke the Von Neumann equation. The derivation of the Von Neumann equation begins with the Schr\"{o}dinger equation and its adjoint:
\begin{align}
\frac{d}{dt}\ket{n} &= -\frac{i}{\hbar}\hat{H}\ket{n}\\
\frac{d}{dt}\bra{n} &= \frac{i}{\hbar}\bra{n}\hat{H}.
\end{align}
To get the von Neumann equation, we take the time derivative of $r$.
\begin{align}
\dot{r} &= \frac{d}{dt}\sum_{n}P_n\ket{n}\bra{n}\nonumber\\
&= \sum_{n}P_n\left(\ket{\dot{n}\bra{n} + \ket{n}\bra{\dot{n}}} \right) \nonumber\\
&= -\frac{i}{\hbar}\sum_{n}P_n\left( \hat{H}\ket{n}\bra{n} - \ket{n}\bra{n}\hat{H}\right) \nonumber\\
&= -\frac{i}{\hbar}(\hat{H}r - r\hat{H})\nonumber\\
&= -\frac{i}{\hbar}[\hat{H},r].
\end{align}
Note that $r$ is an operator, so $\dot{r}$ is expressed in terms of a commutator. The Von Neumann is equivalent to the Schr\"{o}dinger equation. However, instead of describing the time evolution of a wavefunction, the Von Neumann equation describes how the density operator $r$ evolves in time. Now, we have obtained the Von Neumann equation for an atom undergoing transitions due to excitations due to external radiation:
\begin{align}
i\hbar = [\hat{H},r].
\end{align}
However, we are omitting the possibility of spontaneous emission from $\ket{2}$ and $\ket{3}$ (so leaving $\ket{2}$ and $\ket{3}$ only, not necessarily to $\ket{1}$). So, we include additional terms to account for this:
\begin{align}
i\hbar\dot{r} = [\hat{H},r] - \frac{i\hbar}{2}\{\Gamma, r\},
\end{align}
where the curly brackets denote the anti-commutator, and $\Gamma$ is a matrix defined by decay rates $\gamma_n$ from state $\ket{n}$. Note that $\gamma_1 = 0$ because $n=1$ denotes the ground state:
\begin{align}
\Gamma = \begin{pmatrix}
0 & 0 & 0\\
0 & \gamma_2 & 0\\
0 & 0 & \gamma_3
\end{pmatrix}.
\end{align}
In index notation, our Von Neumann equation becomes:
\begin{align}
i\hbar\dot{r}_{ij} = \left( \hat{H}_{ik}r_{kj} - r_{ik}\hat{H}_{kj}\right) - \frac{i\hbar}{2}\left(\Gamma_{ik}r_{kj} + r_{ik}\Gamma_{kj} \right),
\end{align}
where the repeated index $k$ denotes summing over $k$.




	
\end{document}